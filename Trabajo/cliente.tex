

\section{Introducción al cliente en un \textbf{SCV}}

Todos los sistemas de control de versiones se basan en disponer de un  
\textbf{repositorio}, que es el conjunto de información gestionada por
el sistema. Este repositorio contiene el historial de versiones de
todos los elementos gestionados. Con lo cual, lo primero que debe
hacer el cliente es obtener su copia de trabajo del repositorio;
dependiendo de que tipo de \textbf{Sistema de Control de Versiones}
utilice se obtendrá la copia de trabajo de una forma u otra.\\

A partir de este momento ya el cliente podrá realizar sus aportes al
repositorio. Para ello, el cliente trabajara bajo su copia de trabajo
y realizará las modificaciones oportunas. Sin embargo, esas
modificaciones no se tendrán en cuenta en el repositorio hasta que no
se envíen al él. \\

Aunque los tipos de \textbf{SCV} son muy distintos entre ellos, desde
el punto de vista del cliente no lo son tanto: el hecho de que el
repositorio se organice de forma distribuida o centralizada no afecta
al estilo de trabajo del cliente. \\

\subsection{Términos básicos}
\begin{description}
\item [Repositorio] Es el lugar donde se almacenan los datos
  actualizados e históricos en un servidor.
\item [Módulo] Conjunto de directorios y/o archivos dentro del
  repositorio que pertenecen a un proyecto común. 
\item [Rotular (tag)] Darle a alguna versión de cada uno de los
  ficheros del módulo en desarrollo en un momento preciso un nombre
  común ("etiqueta" o "rótulo") para asegurarse de reencontrar ese
  estado de desarrollo posteriormente bajo ese nombre. 
\item [Versión] Una revisión es una versión determinada de un
  archivo. 
\item [Ramificar (branch)] Un módulo puede ser bifurcado en un momento
  de tiempo de forma que, desde ese momento en adelante, dos copias de
  esos ficheros puedan ser desarrolladas a diferentes velocidades o de
  diferentes formas, de modo independiente. 
\item [Desplegar (check-out)] Un despliegue crea una copia de trabajo
  local desde el repositorio. Se puede especificar una revisión
  concreta, y por defecto se suele obtener la última. 
\item [Envío (commit)] Una copia de los cambios hechos a una copia
  local es escrita o integrada sobre repositorio. 
\item [Conflicto] Puede ocurrir en varias circunstancias:
\begin{itemize}
\item Los usuarios X e Y despliegan versiones del archivo A en que las
  líneas n1 hasta n2 son comunes. 
\item El usuario X envía cambios entre las líneas n1 y n2 al archivo A.
\item El usuario Y no actualiza el archivo A tras el envío del usuario X.
\item El usuario Y realiza cambios entre las líneas n1 y n2.
\item El usuario Y intenta posteriormente enviar esos cambios al archivo A.
\end{itemize}
En estos casos el sistema es incapaz de fusionar los cambios y pide al
usuario que sea él quien \textbf{resuelva} el conflicto. 
\item [Resolver] El acto de la intervención del usuario para atender
  un conflicto entre diferentes cambios al mismo documento. 
\item [Cambio (diff)] Un cambio representa una modificación específica
  a un documento bajo control de versiones. 
\item [Lista de Cambios (changelist)] En muchos sistemas de control de
  versiones con commits multi-cambio atómicos, una lista de cambios
  identifica el conjunto de cambios hechos en un único commit. 
\item [Exportación] Una exportación es similar a un check-out, salvo
  porque crea un árbol de directorios limpio sin los metadatos de
  control de versiones presentes en la copia de trabajo. 
\item [Importación] Una importación es la acción de copia un árbol de
  directorios local (que no es en ese momento una copia de trabajo) en
  el repositorio por primera vez.
\item [Fusión (merge)] Una integración o fusión une dos conjuntos de
  cambios sobre un fichero o un conjunto de ficheros en una revisión
  unificada de dicho fichero o ficheros. Puede suceder cuando: 
\begin{itemize}
\item Un usuario, trabajando en esos ficheros, actualiza su copia
  local con los cambios realizados, y añadidos al repositorio, por
  otros usuarios. Análogamente, este mismo proceso puede ocurrir en el
  repositorio cuando un usuario intenta check-in sus cambios. 
\item Después de que el código haya sido branched, y un problema
  anterior al branching sea arreglado en una rama, y se necesite
  incorporar dicho arreglo en la otra. 
\item Después de que los ficheros hayan sido branched, desarrollados
  de forma independiente por un tiempo, y que entonces se haya
  requerido que fueran fundidos de nuevo en un único trunk unificado. 
\end{itemize}
\item [Integración inversa] El proceso de fundir ramas de diferentes
  equipos en el trunk principal del sistema de versiones. 
\item [Actualización (update)] Una actualización integra los cambios
  que han sido hechos en el repositorio (por ejemplo por otras
  personas) en la copia de trabajo local. 
\item [Copia de trabajo] La copia de trabajo es la copia local de los
  ficheros de un repositorio, en un momento del tiempo o revisión
  específicos. Todo el trabajo realizado sobre los ficheros en un
  repositorio se realiza inicialmente sobre una copia de trabajo, de
  ahí su nombre. Conceptualmente, es un cajón de arena o sandbox. 
\item [Congelar] congelar significa permitir los últimos cambios
  (commits) para solucionar las fallas a resolver en una entrega
  (release) y suspender cualquier otro cambio antes de una entrega,
  con el fin de obtener una versión consistente. 
\end{description}

\section{Ciclo de trabajo con \textbf{SCV} centralizado: Subversion}

Cuando un cliente quiere comenzar a desarrollar un proyecto utilizando
Subversion primeramente debe instalar la herramienta de la siguiente
forma: 
\begin{lstlisting}[style=consola]
sudo apt-get install subversion
\end{lstlisting}

Cuando ya tiene la herramienta lista para utilizar ejecutará el
\textit{checkout} inicial:  
\begin{lstlisting}[style=consola]
svn checkout http://ruta/repositorio 
\end{lstlisting}
Por ejemplo: 
\begin{lstlisting}[style=consola]
svn checkout http://192.168.2.101/svn/RepositorioEjemplo 
\end{lstlisting}

A partir de este momento el cliente ya tiene en su poder una copia de
trabajo del repositorio en el cual podrá ejercer sus modificaciones e
incluirlas en el sistema de control de versiones. El ciclo de trabajo
que se debe llevar es el siguiente: 

\begin{itemize}
\item Actualizar la copia de trabajo (por si otros compañeros han
  realizado modificaciones):  
\begin{lstlisting}[style=consola]
svn update
\end{lstlisting}

\item Examinar los cambios a partir de comentarios en las
  revisiones. Cada vez que se envían las modificaciones al servidor se
  incluye un comentario sobre qué es lo que se ha modificado para esa
  revisión. Estos comentarios son los \textit{logs}. Para ver los logs
  se introduce en consola la siguiente línea de comandos: 
\begin{lstlisting}[style=consola]
svn log -r revmenor:revmayor 
\end{lstlisting}

De esta forma se mostrará en pantalla los comentarios de cada revisión
para informarnos de las modificaciones que se han realizado en
ellas. Esto es muy útil ejecutarlo antes de ponernos a modificar para
informarnos de cómo va el proyecto. 

\item Realizar los cambios pertinentes a los ficheros y directorios:
  mover, copiar, borrar, crear... (por ahora serán cambios
  locales). Hay que hacerlos con la asistencia de \textbf{svn}. De
  esta forma se mantendrá el historial de cambios. Se utiliza de la
  siguiente forma: 
\begin{lstlisting}[style=consola]
svn add [fichero|directorio]
svn mkdir directorio
svn delete [fichero|directorio]
svn mv [fichero|directorio]Original [fichero|directorio]Cambiado
svn cp [fichero|directorio]Original [fichero|directorio]Copiado
\end{lstlisting}

Veamos varios ejemplos de uso de estos comandos:
\begin{lstlisting}[style=consola]
svn add fichero1.c                         // creamos fichero1.c
svn add fichero2.c                         // creamos fichero2.c
svn mkdir dirficheritos                    // creamos dirficheritos
svn cp fichero1.c ficheritos/fichero1.h    // copiamos el fichero
svn mv fichero2.c ficheritos/fichero1.cpp  // movemos el fichero
svn delete fichero1.c                      // borramos fichero1.c
\end{lstlisting}

\item Examinar los cambios realizados. Podemos comprobar el estado de
  los ficheros de la copia de trabajo de la siguiente forma: 
\begin{lstlisting}[style=consola]
svn status
\end{lstlisting}

Este comando muestra todos los cambios realizados en la copia de
trabajo desde que se realizó el último \textbf{svn update}. Puede
funcionar sin red ya que sólo es necesario tener las carpetas ocultas
\textit{.svn}. Los códigos más comunes que nos muestra son: 
\begin{itemize}
\item \textbf{A} fichero o directorio añadido.
\item \textbf{D} fichero o directorio borrado.
\item \textbf{U} fichero o directorio actualizado con cambios
  remotos. 
\item \textbf{M} fichero o directorio modificado localmente.
\item \textbf{G} fichero o directorio acutalizado con cambios locales
  y remotos reunidos automáticamente. 
\item \textbf{C} los cambios locales y remotos han de ser reunidos
  localmente (conflicto en modificaciones). 
\item \textbf{?} fichero o directorio que no se haya bajo el sistema
  de control de versiones y no se está ignorando. 
\item \textbf{!} fichero o directorio falta en el sistema de control
  de versiones (puede haberse borrado localmente). 
\item \textbf{~} el fichero o directorio no es del tipo esperado (se
  obtiene un directorio y se esperaba un fichero o viseversa) .
\item \textbf{+} se copiará información de historial además de
  contenido. 
\end{itemize}

\item Examinar las diferencias de ficheros entre dos revisiones. Ésto
  es útil cuando queremos enviar modificaciones cuando no tenemos
  acceso de escritura o para ver que han modificado los
  compañeros. Por defecto compara la revisión modificada localmente
  con la última registrada en el servidor. El comando a utilizar es: 
\begin{lstlisting}[style=consola]
svn diff [-r revmenor:revmayor] fichero
\end{lstlisting}

\item Deshacer posibles cambios en un fichero o directorio:
\begin{lstlisting}[style=consola]
svn revert fichero
\end{lstlisting}
Esto restaura al fichero obtenido al del último \textit{svn update}.

\item Resolver conflictos. Es muy conveniente el realizar un
  \textit{svn update} antes de subir nuestras modificaciones por si
  existiesen conflictos. 

\item Por último, para poder enviar las modificaciones al servidor
  usaremos el comando:
  \begin{lstlisting}[style=consola]
    svn commit -m log\_entre\_comillas
  \end{lstlisting}
  De esta forma quedarán almacenadas en el servidor las modificaciones
  realizadas en nuestra copia de trabajo además de la pequeña
  descripción que aportamos sobre éstas.
\end{itemize}
  

\section{Ciclo de trabajo con \textbf{SCV} distribuido: Git}

Como en todos los sistemas de control de versiones tenemos que
instalar la aplicación. Para \textit{Git} esto se hace con el
siguiente comando:
\begin{lstlisting}[style=consola]
  sudo apt-get install git-core
\end{lstlisting}

Para comenzar con \textit{Git} primero tenemos que introducir nuestros
datos (nombre y correo electrónico) para que el sistema nos indique
como autor en cada una de nuestras revisiones. Además podemos informar
a \textit{Git} de cuál es nuestro editor favorito (por defecto usa
Vim). Realizamos:
\begin{lstlisting}[style=consola]
  git config --global user.name ``Nombre Apellidos''
  git config --global user.email micorreo@ejemplo.com
  git config --global user.editor mieditor
\end{lstlisting}

Para comenzar con el ciclo de trabajo tenemos que crear nuestro
repositorio vacío, o clonar uno existente. También tenemos que tener
en cuenta si podrá acceder al repositorio varios usuarios. Crearemos
un directorio vacío para, a partir de él crear un repositorio vacío
y que se pueda compartir con todos los miembros del grupo al que
pertenezcan los ficheros del repositorio.
\begin{lstlisting}[style=consola]
  mkdir directorioVacio
  git init --shared=group
\end{lstlisting}

Podemos observar que se ha creado un directorio oculto
\textit{.git}. Contiene:
\begin{itemize}
  \item \textbf{config} Fichero con la configuración local de Git para
    este repositorio.
  \item \textbf{description} Descripción del repositorio para
    \textit{gitweb}
  \item \textbf{HEAD} Referencia a la revisión actual con la que
    trabajamos.
  \item \textbf{hooks} Guiones bash o programas para responder ante
    eventos.
  \item \textbf{info/exclude} Fichero del estilo de
    \textit{.gitignore} global para el repositorio y no controlado
    por Git.
  \item \textbf{object} Base de datos de objetos.
  \item \textbf{refs} Referencias por nombre a las puntas de las
    distintas ramas locales y remotas y a otras revisiones etiquetadas
    por nombre. 
\end{itemize}

Para saber el estado de los ficheros en \textit{Git} tenemos el
comando:
\begin{lstlisting}[style=consola]
  git status
\end{lstlisting}

Si en el directorio anterior creamos un fichero f cualquiera y
ejecutamos el comando anterior obtenemos la siguiente salida:
\begin{lstlisting}[style=consola]
  # On branch master
  #
  # Initial commit
  #
  # Untracked files:
  # (use "git add <file>..." to include in what will be committed)
  #
  # f
  nothing added to commit but untracked files present (use "git add" to track)
\end{lstlisting}

Con esto sabemos que:
\begin{itemize}
  \item Nos encontramos en la rama principal del repositorio creado.
  \item La siguiente revisión que creamos será la primera.
  \item Existen algunos fichero cuyos cambios no están monitorizados
    (añadidos al repositorio). En nuestro caso, nos está avisando que
    faltaría añadir el fichero f.
\end{itemize}

Veamos pues cómo podemos añadir ficheros para prepararlos para envío.
\begin{lstlisting}[style=consola]
  git add f
\end{lstlisting}

También podemos borrar ficheros con la siguiente orden:
\begin{lstlisting}[style=consola]
  git rm fichero
\end{lstlisting}

Ahora enviaremos las modificaciones y crearemos la primera revisión
con:
\begin{lstlisting}[style=consola]
  git commit -m comentario\_de\_la\_revision
\end{lstlisting}

Hay que tener en cuenta que con \textit{Git} trabajamos con tres
estructuras de datos: el repositorio (el directorio .git), el índice o
caché (.git/index) y la copia de trabajo (todo lo que está fuera de .git).

Con lo cual podemos comparar qué estados tiene cada directorio y qué
diferencias existen con la siguiente orden:
\begin{lstlisting}[style=consola]
  git diff
\end{lstlisting}

Obteniendo las diferencias entre el directorio de trabajo y la versión
preparada para envío en el índice, además de destacar aquella línea
que no preparemos para envío.

\begin{lstlisting}[style=consola]
  git diff --cached
\end{lstlisting}

Esta orde compara la versión del índice con la versión de la revisión
actual (HEAD), y así nos señalará que versión tenemos preparada para
enviar.

Para poder deshacer cambios en \textit{Git} podemos restaurar un
fichero del directorio de trabajo a la versión que se halle en el
índice de la siguiente forma:
\begin{lstlisting}[style=consola]
  git checkout fichero
\end{lstlisting}

Las formas más comunes de deshacer cambios en un fichero son: \\
\begin{tabular}{c c c}
\hline
Acción a deshacer & Afecta a & Orden \\
\hline
Añadido & & \\ 
Modificado & índice & git reset -- f \\
Borrado & & \\
\hline
Añadido & &  rm f \\
Modificado & dir.trabajo & git checkout f \\
Borrado & & git checkout f \\
\hline
Añadido & & git rm -f f \\
Modificado & ambos & git checkout HEAD f \\
Borrado & & git checkout HEAD f \\
\hline
\end{tabular}

Si no hemos enviado aún nuestros cambios a ningún otro repositorio ni
nadie nos ha tomado de nuestro repositorio podemos corregir errores
fácilmente en la última revisión que hayamos hecho:

\begin{lstlisting}[style=consola]
  git commit --amend
\end{lstlisting}

Con esto la anterior revisión se ignorará y será efectivamente
sustituida por otra que incorporará sus cambios y los que hayamos
hecho ahora.

Para poder etiquetar una revisión ejecutamos:
\begin{lstlisting}[style=consola]
  git tag etiqueta
\end{lstlisting}

También podemos etiquetar cualquier otra revisión creando objetos
reales de tipo ``etiqueta'' con comentarios (-a) y firmada con nuestra
clave pública GnuPG (-s) que tenga el mismo correo y nombre que le
dimos a \textit{Git}:

\begin{lstlisting}[style=consola]
  git tag -a -s etiqueta revision
\end{lstlisting}

Finalmente podemos redistribuir los contenidos de cualquier revisión
mediante la orden :

\begin{lstlisting}[style=consola]
  git archive etiqueta\_de\_la\_revision
\end{lstlisting}


